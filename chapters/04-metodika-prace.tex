\chapter{Metodika práce}

\section{Výber komponentov}
Cieľovú platformu Arduino Uno sme samozrejme museli doplniť ďalšímí komponentmi, ktoré umožňujú realizáciu nášho zadania.

Okrem spomínaného RTC modulu sme pre komunikáciu s používateľom použili 16x2 LCD displej a 4x4 membránovú klávesnicu. Ďalej sme použili piezomenič na vydanie akustického tónu.

\subsection{Zoznam použitých komponentov}
\begin{itemize}
    \item Arduino Uno
    \item RTC modul DS3231
    \item LCD displej LCD1602
    \item membránová klávesnica ER-CCO21601C
    \item piezomenič
    \item potenciometer \SI{10}{\kilo\ohm}
    \item rezistor \SI{220}{\ohm}
\end{itemize}

\section{Návrh zapojenia}
Celkové zapojenie sme navrhli na základe odporúčaných postupov pre jednotlivé komponenty. Vývody na doske Arduino sme volili tak, aby boli súvisiace komponenty zapojené spolu, no v niektorých prípadoch toto nebolo možné vzhľadom na vlastnosti niektorých vývodov (napr. PWM).

Pre realizáciu obvodu sme dočasne využili nepájivé pole. Do budúcnosti je plánované umiestniť hotový výrobok do krabičky.

\section{Implementácia}
Pri programovaní sme použili programovací jazyk C++ a vhodné knižice pre prácu so zvolenými komponentmi.

Po spustení sa inicializuje RTC, pričom ak sa jedná o prvé spustenie po programovaní, tak sa čas kompilácie nastaví ako inicializačný čas, čím odpadá potreba ručne nastavovať čas. Taktiež prebehne inicializácia iných komponentov.

Následne sa v hlavnom cykle programu pravidelne zisťuje aktuálny čas z modulu RTC a porovnáva s časom budenia.
